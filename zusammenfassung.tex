\documentclass[BCOR=5mm,DIV=calc,listof=totoc,headings=big]{scrartcl}
\usepackage{etex}
\usepackage{keyval}
\usepackage[font=footnotesize,labelfont=bf,textfont=normal,format=plain,indention=5em,justification=RaggedRight,labelsep=period]{caption}
\usepackage{dsfont}
\usepackage{stmaryrd}
\usepackage[utf8]{inputenc}
\usepackage[ngerman]{babel}
\usepackage{amsmath}
\usepackage{amssymb}
\usepackage{graphicx}
\usepackage{calc}
\usepackage{fancyhdr}
%\renewcommand*{\chapterpagestyle}{scrheadings}
\usepackage{textcomp}
\usepackage{float}
\usepackage{blindtext}
\usepackage{abstract}
\usepackage{wrapfig}
%\usepackage{floatflt}
%\usepackage{units}
\usepackage{nicefrac}
\usepackage{tikz}
\usepackage{thmbox}
%\usepackage{dcolumn}
\usepackage{subfig}
\usepackage[version=3]{mhchem}
%\usepackage{totpages}
\usepackage{multicol}
\usepackage{multirow}
\usepackage{booktabs}
\usepackage{bibgerm}
\usepackage[T1]{fontenc}
\usepackage{lmodern}
\usepackage[geometry]{ifsym}
%\usepackage{mathtools}
%\usepackage[german]{varioref}
\usepackage[scale=0.8]{geometry}
\usepackage{hyperref}
\usepackage{braket}
\usepackage{commath}
\usepackage{xfrac}
\usepackage{esvect}
\usepackage{siunitx}
\sisetup{decimalsymbol={comma},per=fraction,fraction=sfrac,tophrase=dots,seperr,repeatunits=false}
\usepackage{hepnames}
%\usepackage[Lenny]{fncychap}
\hypersetup{
pdftitle = {Zusammenfassung Experimentalphysik 5},
pdfsubject = {Experimentalphysik 5},
breaklinks = {true},
pdfauthor = {Studenten},
pdfkeywords = {Zusammenfassung,Experimentalphysik,Wintersemester,Uni,Bonn},
colorlinks = {false}
}
\newcommand{\de}{\mathrm{d}}
\newcommand{\abl}[2]{\frac{\mathrm{d} #1}{\mathrm{d} #2}}
\newcommand{\pabl}[2]{\frac{\partial #1}{\partial #2}}
\newcommand{\dabl}[2]{\frac{\delta #1}{\delta #2}}
\newcommand{\arr}[1]{#1_{1}, \ldots, #1_{n}}
\newcommand{\nab}{\vec{\nabla}}
\newcommand{\vecv}{\vec{v}}
\newcommand{\vecr}{\vec{r}}
\newcommand{\lueck}{\hspace{1cm}}
\newcommand{\half}{\frac{1}{2}}
\newcommand{\dive}{\text{div}}
\newcommand{\gradi}{\text{grad}}
\newcommand{\rota}{\text{rot}}
\newcolumntype{d}[1]{D{,}{,}{#1}}


\newcolumntype{,}{D{,}{,}{2}}
\newcolumntype{e}{D{,}{ }{2}}
\newcolumntype{p}[1]{D{.}{,}{#1}}
\newcolumntype{.}{D{.}{,}{2}}
\setcapindent*{1em}

%\setcounter{secnumdepth}{5}
%\setcounter{tocdepth}{4}
\author{Studenten}
\date{Wintersemester 2009/10}
\subject{Experimentalphysik 5} \title{Zusammenfassung}
%\recalctypearea
\begin{document}
%\raggedbottom
%\frontmatter
%\begin{titlepage}
%  \thispagestyle{empty}
\maketitle
%\end{titlepage}
%\tableofcontents
%\mainmatter
\pagenumbering{arabic}
\pagestyle{fancy}
\fancyhf{}
%\fancyhead[L]{\leftmark}
%\fancyhead[R]{\rightmark}
\renewcommand{\headrulewidth}{0.4pt}
\renewcommand{\footrulewidth}{0.4pt}
%\fancyfoot[C]{\thepage}
\setlength{\headheight}{15pt}
\fancyfoot[EL]{\textbf{Seite \thepage} {\scriptsize Studenten}}
\fancyfoot[OR]{{\scriptsize Studenten} \textbf{Seite
\thepage}}
\fancyfoot[OL]{\textbf{Zusammenfassung ExPhy 5}}
\fancyfoot[ER]{\textbf{Zusammenfassung ExPhy 5}}
\fancyhead[EL]{\leftmark}
\fancyhead[ER]{\rightmark}
\fancyhead[OR]{\leftmark}
\fancyhead[OL]{\rightmark}
\fancypagestyle{plain}{
\fancyhf{}
\renewcommand{\headrulewidth}{0.0pt}
\fancyfoot[EL]{\textbf{Seite \thepage} {\scriptsize Studenten}}
\fancyfoot[OR]{{\scriptsize Studenten} \textbf{Seite
\thepage}}
\fancyfoot[OL]{\textbf{Zusammenfassung ExPhy 5}}
\fancyfoot[ER]{\textbf{Zusammenfassung ExPhy 5}}
}
% ----------------------------------------------------------------------------------------------

\section{Radioaktives Zerfallsgesetz}
\label{sec:radi-zerf}
\textbf{Zerfallsgesetz:}
\begin{align*}
  N(t)&=N_{0} \cdot \mathrm{e}^{-\lambda t} &
  \tau&=\frac{1}{\lambda}\\
  t_{\nicefrac{1}{2}}&=\frac{\ln(2)}{\lambda}=\tau \cdot \ln(2)\\
  N\del{t=t_{\nicefrac{1}{2}}}&=\frac{1}{2} \cdot N_{0} \cdot
  \mathrm{e}^{-\lambda t_{\nicefrac{1}{2}}}\\
  \tau: &\text{ mittlere Lebenszeit}\\
  t_{\nicefrac{1}{2}}: &\text{ Halbwertszeit}
\end{align*}
\begin{itemize}
\item Radioaktivität: spontane Eigenemission von Strahlung aus dem Atomkern
\item Herleitung Zerfallsgesetz:
  \begin{align*}
    -\Delta N&=N \lambda \Delta t\\
    \Rightarrow -\de N&=N \lambda \de t\\
    \Rightarrow N(t)&=N_{0} \cdot \mathrm{e}^{-\lambda t}
  \end{align*}
\item Aktivität:
  \begin{align*}
    A&=\lambda N=-\od{N}{t}\\
    \Rightarrow A&=A_{0} \mathrm{e}^{-\lambda t}
  \end{align*}
\item Anzahl Kerne: z.B. $\ce{^{226}Ra}$ $\Rightarrow$ $\frac{Na}{226}$
\end{itemize}

\section{Streuung}
\label{sec:streuung}
\begin{itemize}
\item \textbf{elastisch:} Teilchen vor und nach dem Stoß identisch. Target:
  \begin{itemize}
  \item bleibt im Grundzustand
  \item übernimmt Rückstoßimpuls und kinetische Energie
  \end{itemize}
  BILD1
\item \textbf{inelastisch:} an Target abgegebene Energie regt dieses
  an, nach gewisser Zeit fällt dieses in den Grundzustand zurück
  $\Rightarrow$ Emission\\
  BILD2
\item \textbf{geometrischer Wirkungsquerschnitt:}
  \begin{align*}
    \sigma_{\mathrm{b}}&=\frac{\dot{N}}{\phi_{\mathrm{a}} \cdot
      N_{\mathrm{b}}}=\frac{\text{Zahl der Reaktionen pro
        Zeiteinheit}}{\text{Zahl d. Strahlteilchen pro Zeiteinh.
        pro Flächeneinh.} \cdot \text{Zahl d. Streuzentren}}\\
    \dot{N}&=\phi_{\mathrm{a}}\cdot N_{\mathrm{b}}\cdot\sigma_{\mathrm{b}}\\
    \phi_{\mathrm{a}}&=\frac{\dot{N}_{\mathrm{a}}}{A}=n_{\mathrm{a}}v_{\mathrm{a}}
  \end{align*}
\item \textbf{Wirkungsquerschnitt:}
  \begin{align*}
    \sigma_{\text{tot}}&=\frac{\text{Zahl d. Reaktionen pro
        Zeiteinh.}}{\text{Zahl d. Strahlteilchen pro Zeiteinh.} \cdot
      \text{Zahl der Streuzentren pro Flächeneinh.}}\\
    \sigma_{\text{proton-proton}}&=\SI{40}{\milli\barn} \lueck \SI{10}{\giga\electronvolt}\\
    \sigma_{\text{neutron-neutron}}&=\SI{70}{\femto\barn} \lueck \SI{10}{\giga\electronvolt}
  \end{align*}
\item \textbf{Luminosität:}
  \begin{align*}
    \mathcal{L}&=\phi_{\mathrm{a}} \cdot N_{\mathrm{b}} \cdot
    \dot{N_{\mathrm{a}}} \cdot n_{\mathrm{b}} \cdot d = n_{\mathrm{a}}
    \cdot v_{\mathrm{a}} \cdot N_{\mathrm{b}}
  \end{align*}
  \textbf{Speicherring:}
  \begin{align*}
    \mathcal{L}&=\frac{N_{\mathrm{a}} \cdot N_{\mathrm{b}} \cdot j
      \cdot \nicefrac{v}{u}}{A} & A&=4\pi \sigma_{x} \sigma_{y}
  \end{align*}
  Um eine hohe Luminosität zu erreichen, müssen Strahlen am
  Wechselwirkungs-Punkt auf einen möglichst kleinen Querschnitt
  komprimiert werden.
\item \textbf{Differentieller Wirkungsquerschnitt:}
  \begin{align*}
    \dot{N}\del{\Xi \Theta, \Delta \Omega}&=\mathcal{L}
    \od{\sigma(\Xi, \Theta)}{\Omega}\Delta \Omega\\
    \sigma_{\text{tot}}&>\int_{0}^{E^{\prime}_{\text{max}}}\int_{UIT} \frac{\de
   ^{2}\sigma}{\de\Omega\de E^{\prime}}\de\Omega\de E^{\prime}
  \end{align*}
\end{itemize}

\section{Rutherford-Streuung}
\label{sec:rutherford-streuung}
\begin{itemize}
\item Ziel: Abschätzung für Größe Atomkern
\item Experiment: $\alpha$--Teilchen auf Gold--Platte
\item Streuung auch $\SIrange{160}{170}{\degree}$ $\Rightarrow$
  Ladungen an denen $\alpha$--Teilchen gestreut werden $\ll$ Raum
  verteil als angenommen
\item Ansatz "`Kepler--Problem"' $\Rightarrow$ Coulomb--Kraft als Zentralkraft
\end{itemize}
\textbf{Anmerkung:}\\
Schwerpunkts- und Laborkoordinaten können als identisch betrachtet
werden, wenn die Projektilmasse $\ll$ Stoßpartnermasse\\
$\Rightarrow$ \textbf{Bewegungsgleichung:}
\begin{align*}
  m \ddot{r} - mr\dot{\phi}^{2}&=\frac{1}{4\pi\epsilon_{0}} \frac{Z_{1}Z_{2}e^{2}}{r^{2}}
\end{align*}
BILD3\\
$\Rightarrow$ \textbf{Lösung:}\\
\begin{align*}
  \frac{1}{r}&=\frac{1}{b}\sin\phi +
  \frac{D}{2b^{2}}\del{\cos\phi-1}\\
  D&=\frac{Z_{1}Z_{2}e^{2}}{4\pi\epsilon_{0}T_{\infty}}\\
  b&\text{ Stoßparameter}\\
  \tan{\nicefrac{\theta}{2}}&=\frac{Z_{1}Z_{2}e^{2}}{8\pi\epsilon_{0}bT}\\
  \intertext{Gleichung einer Hyperbel in Polarkoordinaten. Winkel, in
    dem Teilchen gestreut wurden:} \theta&=\pi-\phi\\
  r&\rightarrow \infty\\
  \phi&\rightarrow \pi-\theta\\
  \rightarrow b&=\frac{D}{2}\cot{\nicefrac{\theta}{2}}
\end{align*}
\textbf{Verlauf:}\\
BILD4\\
Kreisring $2\pi b \de b$\\
Öffnungswinkel: $\de \theta=2\pi \sin\theta \de \Omega$
\begin{align*}
  j 2\pi b \de b&=j2\pi \sin\theta \de \sigma (\theta) \de \Omega\\
  \sigma(\theta)&=\frac{b}{\sin\theta} \abs{\od{b}{\theta}}\\
  \Rightarrow \od{\sigma(\theta)}{\Omega}&=\del{\frac{D}{u}}^{2}
  \frac{1}{\sin^{4}\del{\nicefrac{\theta}{2}}}=\del{\frac{Z_{1}Z_{2}e^{2}}{4\pi\epsilon_{0}4
    T_{\infty}}}^{2} \frac{1}{\sin^{4}\del{\nicefrac{\theta}{2}}}
\end{align*}

\section{Relativistische Variablen \& Kinematik}
\label{sec:relat-vari-}
\begin{itemize}
\item 4er--Vektor:
  \begin{align*}
    \begin{pmatrix}
      E \\ \vv{p}
    \end{pmatrix}
  \end{align*}
\item Geschwindigkeit: $\frac{\vv{v}}{c}=\beta=\frac{\vv{p}}{E}$
  $\Rightarrow$ $\gamma=\frac{1}{\sqrt{1-\beta^{2}}}$
\item $t$--Kanal:
  \begin{align*}
    t&=q^{2}=\del{p_{t}-p_{i}}^{2}=-Q^{2}\\
    Q&=\del{m_{\text{eing}}-m_{\text{ausg}}}c^{2}\\
    \Rightarrow \abl{\sigma}{Q^{2}}&=\frac{4\pi
      \del{zZe^{2}}^{2}}{v^{2}Q^{4}}\\
    Q&>0 & \text{exotherm}\\
    Q&<0 & \text{endotherm}\\
    Q&=0 & \text{elastisch}
  \end{align*}
\item 2-Körper--Streuung:
  \begin{align*}
    s&=
  \end{align*}
\end{itemize}
MEHR

\section{Bindungsenergien}
\label{sec:bindungsenergien}
\begin{itemize}
\item Die Gesamtmasse eines Atoms ist geringer als die Gesamtmasse
  seiner Konstituenten
  \begin{align*}
    \Rightarrow m_{\mathrm{k}}(Z,N)&=Z \cdot m_{\mathrm{p}} + N \cdot
    m_{\mathrm{n}} - \Delta m_{\mathrm{k}} & \text{Massendefekt}
  \end{align*}
\item es muss Energie aufgewandt werden, um die Konstituenten zu
  trennen $\Rightarrow$ Bindungsenergie\\
HIER KOMMEN WILDE FORMELN
\end{itemize}

\section{Tröpfchenmodell}
\label{sec:tropfchenmodell}
\begin{itemize}
\item Form so, dass $E_{\text{ges}}$ minimal; Minimierung der
  Oberflächenspannung, kurzreichweitige Kraft
\item große Zahl an Nukleonen in schweren Kernen
\item Wert der Bindungsenergie steigt mit Zahl der Nukleonen im Kern
\item Messung: Sättigung bei $\SIrange{55}{60}{\amu}$ $\Rightarrow$ kurzreichweitig
\item nicht alle Nukleonen im Kern erfahren gleiche Kraft:\\
  die an Oberfläche $\Rightarrow$ weniger Nachbarn $\Rightarrow$
  kleiner Kraft\\
  \textbf{Coulomb:} lockert Bindung; Asymmetrie Zahl Proton \& Neutron
  mindert Bindungsenergie\\
  Paarungskräfte zwischen gleichartigen Nukleonen verstärken Bindung
\end{itemize}

\section{Weizsäcker--Massenformel}
\label{sec:weizs-mass}
FORMELKRAM

\section{Fermi-Gas--Modell}
\label{sec:fermi-gas-modell}
\begin{itemize}
\item Nukleonen befinden sich in einem statistischen Ensemble
\item Nukleonen sind Fermionen (Spin = $\nicefrac{1}{2}$)
\item Gas unabhängiger Teilchen
\item Nukleonen befinden sich in einem Kastenpotential
\end{itemize}
FORMELZEUGS

\section{Photoeffekt}
\label{sec:photoeffekt}

\section{Compton--Effekt}
\label{sec:compton-effekt}

\section{Paarbildung}
\label{sec:paarbildung}

\section{Ionistationsprozesse geladener Teilchen}
\label{sec:ionist-gelad-teilch}
FORMELZEUGS

\begin{itemize}
\item Maximum Ionisationsdichte Bragg--Peak
\item \textbf{Elektronen \& Positronen}
  \begin{itemize}
  \item kurze Wechselwirkungszeit $\Rightarrow$ können nur minimal ionisieren
  \item relativistischer Energiebereich: Cerenkov--Effekt
  \end{itemize}
\item \textbf{Cerenkov--Effekt}
  \begin{itemize}
  \item langsames \Pelectron polarisiert beim Durchgang Atome
  \item da symmetrisches Polarisationsfeld $\Rightarrow$ keine EM--Emission
  \item schnelles \Pelectron "`überholt"' sein Strahlungsfeld
  \item Dipolfeld entsteht $\Rightarrow$ Cerenkov--Strahlung
  \end{itemize}
\item \textbf{Bremsstrahlung}\\
  \Pelectron strahlt Photonen ab, beim Abbremsen im Feld des Kerns
  linear mit Energie, quadratisch mit $Z$
\end{itemize}

\section{Driftkammer}
\label{sec:driftkammer}
\begin{itemize}
\item angelegte Hochspannung $\Rightarrow$ elektrisches Feld
\item geladenes Teilchen ionisiert Gas auf dem Weg durch die Driftkammer
\item \Pelectron driften entlang der Feldlinien
\item lineare Beziehung:
  \begin{align*}
    x&=x_{\text{Draht}}+v_{\text{Drift}}\del{t_{\text{Draht}}-t_{0}}
  \end{align*}
\end{itemize}
werden auch Proportionalitätszähler genannt. 

\section{Gas-Ionisations--Detektor \emph{(s.o.)}}
\label{sec:gas-ionis-detekt}
\begin{itemize}
\item Feldstärke Anodendrähte
  \begin{align*}
    E&=\frac{1}{r} \frac{V_{0}}{\ln\del{\frac{a}{b}}}\\
    a&=\phi \text{ Kathode}\\
    b&=\phi \text{ Anode}
  \end{align*}
\item verschiedene Betriebsmodi:\\
  u.A. vollständige Entladung: Geiger-Müller
\end{itemize}

\section{Szintillatoren}
\label{sec:szintillatoren}

\section{Photomultiplier}
\label{sec:photomultiplier}

\section{Halbleiterdetektoren}
\label{sec:halbleiterdetektoren}
\begin{itemize}
\item Vorteile:\\
  kleiner dimensioniert: Energieabgabe erfolgt kleinstufiger\\
  $\Rightarrow$ höhere Auflösung: größere Reichweite
\item \Pproton-\Pneutron--Übergang: wechselseitige Diffusion von
  \Pelectron und \Pproton
  \begin{itemize}
  \item Diffusionsspannung $V_{0}$
  \item Verarmungsschicht (frei von beweglichen Ladungsträgern)
  \end{itemize}
\item eintretende ionisierende Teilchen (in Sperrschicht) erzeugen
  \Pelectron-Lock--Paar (\Pelectron im Valenzband werden in CB angeregt)
\item Prinzip entspricht Festkörper Ionisationskammer
\end{itemize}

\section{$\gamma$--Spektren}
\label{sec:gamma-spektren}
\begin{itemize}
\item höchste Peaks: Totalabsorption
\item nach Locke: Compton--Kontinuum
\item Compton--Kante $\hat{=}$ Streuwinkel $\SI{180}{degree}$
\item auf Compton--Kontinuum: Rückstreupeak\\
  (tritt bei Messungen mit radioaktiven Quellen aus Streuprozessen mit
  Materialien auf an Quellen--Rückseite)
  \begin{itemize}
  \item $E_{\mathrm{R}}-E_{\mathrm{P}}-E_{\text{Compton--Kante}}$
  \end{itemize}
\item ist $\gamma$--Energie $>$ Paarerzeugungsschwelle ($\sim \SI{1}{\MeV}$)
  \begin{itemize}
  \item 2 $\gamma$--Quanten aus Positronenzerstrahlung bei $\SI{511}{\keV}$
  \end{itemize}
\item single-- / double--escape--peaks: Abstand von
  $m_{\mathrm{e}}c^{2}$, $2 \cdot m_{\mathrm{e}}c^{2}$ falls
  $\gamma$--Quanten nicht vollständig absorbiert werden
\end{itemize}

\section{Multipolentwicklung}
\label{sec:multipolentwicklung}
FORMELKRAM 

\end{document}